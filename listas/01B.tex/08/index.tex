\item
  De princípio, definimos:
  \begin{itemize}
    \item $H$: HIV positivo

    \item $\overline{H}$: HIV negativo
    \item $T$: teste indica HIV positivo
    \item $\overline{T}$: teste indica HIV positivo
  \end{itemize}

  Pelo enunciado, temos:
  \[\begin{array}{l}
  p(H) = 0.08\\
  p(\overline{H}) = 1 - p(H) = 1 - 0.08 = 0.92\\
  p(T|H) = 0.98\\
  p(\overline{T}|H) = 1-p(T|H) = 0.02\\
  p(T|\overline{H}) = 0.03\\
  p(\overline{T}|\overline{H}) = 1-p(T|\overline{H}) = 0.97
  \end{array}\]

  \begin{enumerate}
    \item
    Utilizando o Teorema de Bayes.
    \[\begin{array}{l}
    p(H|T) = \frac{{p(T|H)p(H)}}{{p(T|H)p(H) + p(T|\overline{H})p(\overline{H})}}\\
     = \frac{{0.98 \cdot 0.08}}{{0.98 \cdot 0.08 + 0.03 \cdot 0.92}} \approx 0.739
    \end{array}\]

    \item
    \[p(\overline{H}|T) = 1 - p(H|T) \approx 1 - 0.739 = 0.261\]

    \item
    Utilizando o Teorema de Bayes.
    \[\begin{array}{l}
    p(H|\overline {T}) = \frac{{p(\overline {T}|H)p(H)}}{{p(\overline {T}|H)p(H) + p(\overline {T}|\overline {H})p(\overline {H})}}\\
     = \frac{{0.02 \cdot 0.08}}{{0.02 \cdot 0.08 + 0.97 \cdot 0.92}} \approx 0.002
    \end{array}\]

    \item
    \[p(\overline{H}|\overline{T}) = 1 - p(H|\overline{T}) \approx 1 - 0.002 = 0.998\]

  \end{enumerate}
