\item
O valor esperado para o primeiro dado jogado, tal que $X$ é a variável aleatória que representa o valor do resultado.

$\begin{array}{l}
p({X=x_i}) = \frac{1}{6}; \forall x_i \in \{1, \dots, 6 \}\\
E(X) = \sum\nolimits_{i = 1}^6 {p({X=x_i}) \cdot X(s)} \\
= \frac{1}{6}\left( 1+2+3+4+5+6 \right) = \frac{7}{2}
\end{array}$

A esperança para a soma de dois dados jogados, denotado pela variável $Y$, $X_1$ e $X_2$ a variável aleatória representando o resultado de cada dado, é:

$\begin{array}{l}
E(Y) = E(X_1 + X_2) = E(X_1) + E(X_2) = \frac{7}{2} + \frac{7}{2} = 7
\end{array}$

Criando uma tabela $X_1$ e $X_2$.

$\begin{array}{l}
(1, 1) \Rightarrow X = 1 \rm{\ e\ } Y = 2 \Rightarrow XY = 2\\
(1, 2) \Rightarrow X = 1 \rm{\ e\ } Y = 3 \Rightarrow XY = 3\\
(1, 3) \Rightarrow X = 1 \rm{\ e\ } Y = 4 \Rightarrow XY = 4\\
(1, 4) \Rightarrow X = 1 \rm{\ e\ } Y = 5 \Rightarrow XY = 5\\
(1, 5) \Rightarrow X = 1 \rm{\ e\ } Y = 6 \Rightarrow XY = 6\\
(1, 6) \Rightarrow X = 1 \rm{\ e\ } Y = 7 \Rightarrow XY = 7\\
(2, 1) \Rightarrow X = 2 \rm{\ e\ } Y = 3 \Rightarrow XY = 6\\
\dots \\
XY \rightarrow f(x_1, x_2) = x_1 \cdot (x_1+x_2) = x_1^2 = x-1+x_2
\end{array}$

Logo, calculando $E(XY)$, temos:

$\begin{array}{l}
p(XY)\\
p(2) = 1/36 \\
p(3) = 1/36 \\
p(4) = 1/36 \\
p(5) = 1/36 \\
p(6) = 2/36 \\
p(7) = 1/36 \\
p(8) = 1/36 \\
p(10) = 1/36 \\
p(12) = 2/36 \\
p(14) = 1/36 \\
p(15) = 1/36 \\
p(16) = 1/36 \\
p(18) = 1/36 \\
p(20) = 1/36 \\
p(21) = 1/36 \\
p(24) = 2/36 \\
p(27) = 1/36 \\
p(28) = 1/36 \\
p(30) = 1/36 \\
p(32) = 1/36 \\
p(35) = 1/36 \\
p(36) = 1/36 \\
p(40) = 2/36 \\
p(42) = 1/36 \\
p(45) = 1/36 \\
p(48) = 1/36 \\
p(50) = 1/36 \\
p(54) = 1/36 \\
p(55) = 1/36 \\
p(60) = 1/36 \\
p(66) = 1/36 \\
p(72) = 1/36 \\
E(XY) = \sum\nolimits_{}{p(XY) \cdot XY} = 27.4166666667
\end{array}$

Portanto, $E(XY) \neq E(X)E(Y) = \frac{7}{2} \cdot 7 = 24.5$.

% $\begin{array}{l}
% E(XY) = \sum\nolimits_{x_1 \in \{1, \cdots, 6\}} {\sum\nolimits_{x_y \in \{1, \cdots, 6\}{p() \cdot X(a_i)}} \\
% \end{array}$
