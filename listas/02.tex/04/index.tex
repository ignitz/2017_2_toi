\item
	Exercise 3.11.[2, p.61] Mrs S is found stabbed in her family garden. Mr S
	behaves strangely after her death and is considered as a suspect. On
	investigation of police and social records it is found that Mr S had beaten
	up his wife on at least nine previous occasions. The prosecution advances
	this data as evidence in favour of the hypothesis that Mr S is guilty of the
	murder. `Ah no,' says Mr S's highly paid lawyer, `\textit{statistically}, only one
	in a thousand wife-beaters actually goes on to murder his wife.1 So the
	wife-beating is not strong evidence at all. In fact, given the wife-beating
	evidence alone, it's extremely \textit{unlikely} that he would be the murderer of
	his wife - only a $1/1000$ chance. You should therefore nd him innocent.'
	Is the lawyer right to imply that the history of wife-beating does not
	point to Mr S's being the murderer? Or is the lawyer a slimy trickster?
	If the latter, what is wrong with his argument?
	[Having received an indignant letter from a lawyer about the preceding
	paragraph, I'd like to add an extra inference exercise at this point: \textit{Does
	my suggestion that Mr. S.'s lawyer may have been a slimy trickster imply
	that I believe all lawyers are slimy tricksters?} (Answer: No.)]
