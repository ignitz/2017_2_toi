\item Answer formally the following questions:
\begin{enumerate}
  \item Describe succinctly the two most common interpretations of probability: the \textit{frequentist} interpretation, and the \textit{Bayesian interpretation}.
  \item Define the concepts of \textit{forward probability} and of \textit{inverse probability}.
  \item What is the difference between the terms \textit{likelihood} and \textit{probability}? In what situation should each of them be used?
\end{enumerate}

\subsection*{Resposta}

\begin{enumerate}
  \item
  \textbf{Interpretação frequentista}
  \begin{itemize}
    \item Se jogarmos uma moeda 100 vezes e der como resultado cara 55 vezes.
    Então a probabilidade estimada é 0.55.
    \item A cada 1.2 milhão de voos de avião, um deles sofre acidente.
    Então a probabilidade estimada é $\frac{1}{12 \cdot 10^5}$.
  \end{itemize}

  \textbf{Interpretação Bayesiana}

  \begin{itemize}
    \item Se eu tenho certeza que a moeda é justa (não-viciada). A minha "crença"
    de que a probabilidade da moeda dar cara se jogada uma vez é $0.5$.

    \item Dado que sei que a moeda possui cara nas duas faces. A probabilidade de
    dar resultado cara é $1$. (Atualizar uma probabilidade posterior).
  \end{itemize}

  \item
  \textbf{Forward probability} descreve um modelo geral de um processo assumindo
  certos dados.
  % problems involve a generative model that describes a
  % process that is assumed to give rise to some data.
  % The task is to compute the probability distribution or expectation of some
  % quantity that depends on the data.

  \item
  \textbf{inverse probability} descreve um modelo em que é calculado a
  probabilidade   condicional de um ou mais variáveis não observadas no
  processo, dado tais variáveis observadas.
  % problems involve a generative model of a process.
  % But instead of computing the probability distribution of some quantity
  % produced by the process, we compute the conditional probability of one or
  % more of the unobserved variables in the process, given the observed variables.

\end{enumerate}
